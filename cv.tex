%-------------------------
% Rover Resume - Fancy Template
% Link: https://github.com/subidit/rover-resume
%------------------------

\documentclass[11pt]{article}

\usepackage[T1]{fontenc}
\usepackage{inter} % https://tug.org/FontCatalogue/
\renewcommand*\familydefault{\sfdefault}

\usepackage{verbatim} % comments

\usepackage{accsupp} % fact Right-Mouse-Button (RMB) copy

\usepackage{geometry}
\geometry{
a4paper,
top=1.8cm,
bottom=1in,
left=2.5cm,
right=2.5cm
}

\setcounter{secnumdepth}{0} % remove section numbering
\pdfgentounicode=1 % make ATS friendly

\usepackage{enumitem}
\setlist[itemize]{
    noitemsep,
    left=0pt..1.5em
}
\setlist[description]{itemsep=0pt}
\setlist[enumerate]{align=left}

\usepackage[dvipsnames]{xcolor}
% \usepackage[dvipsnames, svgnames, x11names]{xcolor} 
% \usepackage[dvipsnames]{xcolor} % xcolor.pdf Sec.4 Colors by Name
\colorlet{icnclr}{gray}
% \colorlet [⟨type⟩]{⟨name⟩}[⟨num model⟩]{⟨color ⟩}
% \definecolor[⟨type⟩]{⟨name⟩}{⟨model-list⟩}{⟨spec-list⟩}


\usepackage{titlesec}
% \titlespacing{command}{left spacing}{before spacing}{after spacing}[right]
% \titlespacing{\section}{0pt}{*3}{*1}
\titlespacing{\subsection}{0pt}{*0}{*0}
\titlespacing{\subsubsection}{0pt}{*0}{*0}
% \titleformat{<command>}[<shape>]{<format>}{<label>}{<sec>}{<before-code>}[<after-code>]  
\titleformat{\section}{\color{Sepia}\large\fontseries{black}\selectfont\uppercase}{}{}{\ruleafter}[\global\RemVStrue]
\titleformat{\subsection}{\large\fontseries{semibold}\selectfont}{}{}{\rvs}
\titleformat{\subsubsection}{\large\fontseries{medium}\selectfont}{}{}{}

\usepackage{xhfill} 
\newcommand\ruleafter[1]{#1~\xrfill[.5ex]{1pt}[gray]} % add rule after title in .5 x-height 

\newif\ifRemVS % remove vspace between \section & \subsection
\newcommand{\rvs}{
    \ifRemVS
        \vspace{-1.5ex}
    \fi
    \global\RemVSfalse
}


\usepackage{fontawesome5}

\usepackage[bookmarks=false]{hyperref} % [imp!]
\hypersetup{ % https://en.wikibooks.org/wiki/LaTeX/Hyperlinks
    colorlinks=true,
    urlcolor=Sepia,
    pdftitle={Pavel Kovalenko Resume},
}

\usepackage[page]{totalcount}
\usepackage{fancyhdr}
\pagestyle{fancy}
\renewcommand{\headrulewidth}{0pt}	
\fancyhf{}							
\cfoot{\color{darkgray} Pavel Kovalenko -- Page \thepage{} of \totalpages}

\begin{document}

%== HEADER ==%
\begin{center}
    {\fontsize{36}{36}\selectfont\interthin PAVEL\interheavy KOVALENKO} \\ \bigskip
    {\fontsize{25}{25}\selectfont\interthin Junior Java Software Developer} \\ \bigskip
    {\color{icnclr}\faEnvelope[regular]} \href{mailto:paaavel.2024@gmail.com}{paaavel.2024@gmail.com} $|$ 
    {\color{icnclr}\faIcon{mobile-alt}} 89002890589 $|$
    {\color{icnclr}\faGithub} \href{https://github.com/PaaavelZ}{PaaavelZ} $|$
    {\color{icnclr}\faMapMarker} Saint-Petersburg
\end{center}

\section{Education}
%==============
\subsection{\href{https://www.spbstu.ru/}{\underline{SPbSTU}} $|$ {\normalfont\textit{\href{https://www.spbstu.ru/structure/graduate_school_software_engineering/}{\underline{Bachelor of Software Engineering}}}} \hfill 2020 - 2024}
\begin{itemize}
    \item \textbf{GPA}: 4.7
    \item \textbf{What was studied}: DBMS, Microservice Architecture, Client-Server Applications, Programming of Microcontrollers, Networks and Telecommunications, System Design, Machine Learning \& Neural Networks, Project Management
    \item \textbf{Stack}: Java, PostgreSQL, Spring (Boot, Web, Data JPA, Security), REST, Docker/Docker Compose, Python, C++, Matlab
\end{itemize}

\subsection{\href{https://polis.vk.company/}{\underline{Technopolis} $|$ {\normalfont\textit{\underline{Java-developer of high-load applications}}}} \hfill 2021 - 2023}
\begin{itemize}
    \item \textbf{GPA}: 4.6
    \item \textbf{What was studied}: SQL/NoSQL, HighLoad, Application Profiling, Machine Learning, Mobile Development, Basic Frontend (HTML, CSS, JS/TS), Information Security \& Blockchain, Testing
    \item \textbf{Stack}: Java, MS SQL, Apache Cassandra, Apache Spark, Apache Zeppelin, Apache Kafka, Apache Airflow, Hadoop, Scala, Solidity, Selenium/Selenide, Espresso/Kaspresso, Kotlin
\end{itemize}

\section{Experience (total - 1 year 6 months)}
%===================
\subsection{\href{https://dzen.ru}{\underline{Dzen}} $|$ {\normalfont\textit{Java Developer Intern}} \hfill March 2023 -- July 2023 (4 months)}
\subsubsection{\underline{\textbf{Stack}: Java, YQL, YDB, Yandex.Monitoring, \href{https://habr.com/ru/companies/yandex/articles/520134/}{Apphost}, Python, Jira / Yandex.Tracker}}
\subsubsection{I developed a news feed for \href{https://dzen.ru}{\underline{dzen.ru}}. In more detail what did I do:}
\begin{itemize}
    \item I created a \textbf{monitoring and alert system} for the service to monitor its performance
    \item I conducted several \textbf{A/B experiments} that have shown a lack of demand for some functionality from users
    \item I added some \textbf{offheap fields} and sent them through several services working with \textbf{APIs} and changing them 
    \item I implemented a \textbf{container of widgets} (weather, geoposition, rates of currencies, etc.) for simplifying the adding of new ones
    \item I also wrote \textbf{automatic Python scripts} to refactor a huge amount of similar code (6k+ lines)
    \item In addition to the routine task flow, I \textbf{created a documentation} about monitoring systems, entities and its communication in Dzen at all that my colleagues used
    \item Sometimes I \textbf{organized the entire project team} in Scrum StandUps
\end{itemize}

\subsection{\href{https://polis.vk.company/}{\underline{Technopolis}} $|$ {\normalfont\textit{Java Developer}} \hfill Feb 2023 -- June 2023 (5 months)}
\subsubsection{\underline{\textbf{Stack}: Java, Spring (Boot), PostgreSQL, TamTam API, \href{https://robbi.ai/}{robbi.ai}}}
\subsubsection{I developed \href{https://github.com/Sanerins/tamtam-one-coffee-bot}{\underline{a graduation project <<RandomCoffee Bot>>}} in the team according to \href{https://ok.ru/}{\underline{the Odnoklassniki}} business case to increase social communication within the company. In more detail what did I do:}
\begin{itemize}
    \item I created \textbf{a PostgreSQL database class interface} to organize and access all application data
    \item I added the ability to \textbf{save the history of user connections} to improve the quality of the search for interlocutors
    \item I \textbf{refactored a code} to improve its readability
\end{itemize}

\subsection{\href{https://www.spbstu.ru/}{\underline{SPbSTU}} $|$ {\normalfont\textit{Python Developer}} \hfill Feb 2022 -- June 2022 (5 months)}
\subsubsection{\underline{\textbf{Stack}: Python (PIL), SQLite,  \href{https://github.com/python-telegram-bot/python-telegram-bot}{Telegram API}, Docker}}
\subsubsection{I developed \href{https://github.com/PaaavelZ/FPA-pybot}{\underline{a Telegram bot}} in the team according to \href{https://www.wbc-c.ru/}{\underline{the Belorusskaya Cosmetika}} business case to protect photos from plagiarism in the Internet. In more detail what did I do:}
\begin{itemize}
    \item I created \textbf{a SQLite database class interfaces} to organize and access all application data
    \item I added \textbf{the processing of photo archives}
    \item I \textbf{prepared the documentation} for the project
    \item I \textbf{interacted with the customer} and looked for a common language to solve problems with him
\end{itemize}

\subsection{\href{https://www.spbstu.ru/}{\underline{SPbSTU-based factory}} $|$ {\normalfont\textit{Java Developer}} \hfill Oct 2021 -- Jan 2022 (4 months)}
\subsubsection{\underline{\textbf{Stack}: Java, Spring (Boot, Web, Data JPA, Security), HTML, CSS, JS, \href{https://deepsource.com/}{DeepSource}}}
\subsubsection{I developed \href{https://github.com/martyn-fanclub/tracking-system}{\underline{a Tracking System}} in the team according to \href{https://www.spbstu.ru/}{\underline{SPbSTU-based factory}} business case to track the progress of work on the machines by the employees of the enterprise. In more detail what did I do:}
\begin{itemize}
    \item I implemented a \textbf{global timer} to see the current time in UI
    \item I implemented \textbf{some endpoints} of the system server
    \item I \textbf{interacted with the customer} and looked for a common language to solve problems with him
\end{itemize}

\section{Projects}
%=================
\subsection{\href{https://polis.vk.company/}{\underline{Technopolis}} $|$ {\normalfont\textit{Java Developer}} \hfill Feb 2022 -- Dec 2022}
\subsubsection{\underline{\textbf{Stack}: Java, Cassandra, Lua, JUnit, \href{https://github.com/odnoklassniki/one-nio}{one-nio}, \href{https://github.com/giltene/wrk2}{wrk2}, \href{https://github.com/async-profiler/async-profiler}{async-profiler}}}
\subsubsection{I developed \href{https://github.com/polis-vk/2022-highload-dht/tree/main/src/main/java/ok/dht/test/kovalenko}{\underline{my own NoSQL DB}} by myself as part of \href{https://education.vk.company/}{\underline{the VK educational project}}, Cassandra was taken as a prototype. In more detail what features did I do:}
\begin{itemize}
    \item As a basic part of DB, I added \textbf{Persistency} to save data, \textbf{Range queries} with binary search and \textbf{Compaction} to remove old data
    \item As a part of network communication with DB, I added \textbf{HTTP REST API} to get data by endpoints, \textbf{Asynchronous server} to unload SelectorThreads, \textbf{Sharding} to up horizontal scalability, \textbf{Replication} to ensure fault tolerance, \textbf{Inner Asynchronous communication} within the cluster nodes, \textbf{Range queries} with HTTP Chunked Response
    \item Also I profiled my DB and tested its performance actively, and got detailed feedback from experts after each feature
\end{itemize}

\begin{comment}
\subsection{\href{https://github.com/PaaavelZ/tinkoff_bot}{\underline{Pet-project}} $|$ {\normalfont\textit{Python Developer}} \hfill June 2022 -- July 2022}
\subsubsection{\underline{\textbf{Stack}: Python, \href{https://github.com/python-telegram-bot/python-telegram-bot}{Telegram API}, SQLite, \href{https://github.com/Tinkoff/invest-python}{Tinkoff API}}}
\subsubsection{I developed \href{https://github.com/PaaavelZ/tinkoff_bot}{\underline{a Telegram financial bot}} for personal use. In more detail what did I do:}
\begin{itemize}
    \item I created \textbf{a SQLite database class interface} to organize and access all application data
    \item I implemented \textbf{the forming and sending aggregated reports} of \href{https://www.tinkoff.ru/invest/}{\underline{the Tinkoff Invest's}} daily trade statistics for chosen accounts as well as the Tinkoff Investment App but it is available only for a month officially
    \item I also added the \textbf{subscription's capabilities} to be able to independently write to the bot and receive personal reports from it
\end{itemize}
\end{comment}

\section{Skills}
%===============
\begin{description}
    \item[Technical] Java, Spring (Boot, Data JPA, Web, Security), PostgreSQL, Cassandra, Docker, Python, C++, Yandex.Monitoring, Jira
    \item[Language] English - B1
\end{description}

\section{Reading List}
%===============
\begin{description}
    \item \href{https://www.amazon.com/Clean-Code-Handbook-Software-Craftsmanship/dp/0132350882}{Robert C. Martin <<Clean Code>>}
\end{description}

\section{Certification \& Awards}
%===============================
\begin{enumerate}[itemsep=0pt]
    \item [2024] \href{https://disk.yandex.ru/i/_CuHJW-1k6qG8A}{\underline{EF SET English B1 Certificate}}
    \item [2023] \href{https://disk.yandex.ru/i/Ftcyy4kuYyXlEQ}{\underline{Technopolis Graduation Certificate}}
\end{enumerate}

\end{document}
