%-------------------------
% Rover Resume - Fancy Template
% Link: https://github.com/subidit/rover-resume
%------------------------

\documentclass[11pt]{article}

\usepackage[T1]{fontenc}
\usepackage{inter} % https://tug.org/FontCatalogue/
%\renewcommand*\familydefault{\sfdefault}

\usepackage[utf8]{inputenc}
\usepackage[english, russian,]{babel} % ru lang

\usepackage{verbatim} % comments

\usepackage{geometry}
\geometry{
a4paper,
top=1.8cm,
bottom=1in,
left=2.5cm,
right=2.5cm
}

\setcounter{secnumdepth}{0} % remove section numbering
\pdfgentounicode=1 % make ATS friendly

\usepackage{enumitem}
\setlist[itemize]{
    noitemsep,
    left=0pt..1.5em
}
\setlist[description]{itemsep=0pt}
\setlist[enumerate]{align=left}

\usepackage[dvipsnames]{xcolor}
% \usepackage[dvipsnames, svgnames, x11names]{xcolor} 
% \usepackage[dvipsnames]{xcolor} % xcolor.pdf Sec.4 Colors by Name
\colorlet{icnclr}{gray}
% \colorlet [⟨type⟩]{⟨name⟩}[⟨num model⟩]{⟨color ⟩}
% \definecolor[⟨type⟩]{⟨name⟩}{⟨model-list⟩}{⟨spec-list⟩}


\usepackage{titlesec}
% \titlespacing{command}{left spacing}{before spacing}{after spacing}[right]
% \titlespacing{\section}{0pt}{*3}{*1}
\titlespacing{\subsection}{0pt}{*0}{*0}
\titlespacing{\subsubsection}{0pt}{*0}{*0}
% \titleformat{<command>}[<shape>]{<format>}{<label>}{<sec>}{<before-code>}[<after-code>]  
%\titleformat{\section}{\color{Sepia}\large\fontseries{black}\selectfont\uppercase}{}{}{\ruleafter}[\global\RemVStrue] % A PROBLEM!
\usepackage[koi8-r]{inputenc}
\titleformat{\section}{\color{Sepia}\large\fontseries{black}\selectfont\uppercase}{}{}{\ruleafter}[\global\RemVStrue]
\titleformat{\subsection}{\large\fontseries{semibold}\selectfont}{}{}{\rvs}
\titleformat{\subsubsection}{\large\fontseries{medium}\selectfont}{}{}{}

\usepackage{xhfill} 
\newcommand\ruleafter[1]{#1~\xrfill[.5ex]{1pt}[gray]} % add rule after title in .5 x-height 

\newif\ifRemVS % remove vspace between \section & \subsection
\newcommand{\rvs}{
    \ifRemVS
        \vspace{-1.5ex}
    \fi
    \global\RemVSfalse
}


\usepackage{fontawesome5}

\usepackage[bookmarks=false]{hyperref} % [imp!]
\hypersetup{ % https://en.wikibooks.org/wiki/LaTeX/Hyperlinks
    colorlinks=true,
    urlcolor=Sepia,
    pdftitle={Павел Коваленко Резюме},
}

\usepackage[page]{totalcount}
\usepackage{fancyhdr}
\pagestyle{fancy}
\renewcommand{\headrulewidth}{0pt}	
\fancyhf{}							
\cfoot{\color{darkgray} Павел Коваленко -- Страница \thepage{} из \totalpages}

\begin{document}

%== HEADER ==%
\begin{center}
    {\fontsize{36}{36}\selectfont\interthin ПАВЕЛ \interheavy КОВАЛЕНКО} \\ \bigskip
    {\fontsize{25}{25}\selectfont\interthin Junior Java Software Developer} \\ \bigskip
    {\color{icnclr}\faEnvelope[regular]} \href{mailto:paaavel.2024@gmail.com}{paaavel.2024@gmail.com} $|$ 
    {\color{icnclr}\faIcon{mobile-alt}} 89002890589 $|$
    {\color{icnclr}\faGithub} \href{https://github.com/PaaavelZ}{PaaavelZ} $|$
    {\color{icnclr}\faMapMarker} Санкт-Петербург
\end{center}

\section{EDUCATION}
%==============
\subsection{\href{https://www.spbstu.ru/}{\underline{СПбПУ}} $|$ {\normalfont\textit{\href{https://www.spbstu.ru/structure/graduate_school_software_engineering/}{\underline{Программная инженерия}}}} \hfill 2020 - 2024}
\begin{itemize}
    \item \textbf{Средний балл}: 4.7
    \item \textbf{Что изучалось}: СУБД, микросервисная архитектура, клиент-серверные приложения, программирование микроконтроллеров, сети и телекоммуникации, системный дизайн, ML, управление проектом
    \item \textbf{Стек}: Java, PostgreSQL, Spring (Boot, Web, Data JPA, Security), REST, Docker/Docker Compose, Python, C++, Matlab
\end{itemize}

\subsection{\href{https://polis.vk.company/}{\underline{Технополис} $|$ {\normalfont\textit{\underline{Java-разработчик высоконагруженных приложений}}}} \hfill 2021 - 2023}
\begin{itemize}
    \item \textbf{Средний балл}: 4.6
    \item \textbf{Что изучалось}: SQL/NoSQL, HighLoad, профилирование, ML, мобильная разработка, фронтенд (HTML, CSS, JS/TS), ИБ и блокчейн, тестирование
    \item \textbf{Stack}: Java, MS SQL, Apache Cassandra, Apache Spark, Apache Zeppelin, Apache Kafka, Apache Airflow, Hadoop, Scala, Solidity, Selenium/Selenide, Espresso/Kaspresso, Kotlin
\end{itemize}

\section{EXPERIENCE (TOTAL - 1 year 3 months)}
%===================
\subsection{\href{https://dzen.ru}{\underline{Дзен}} $|$ {\normalfont\textit{Java Developer Intern}} \hfill Март 2023 -- Июль 2023 (4 месяца)}
\subsubsection{\underline{\textbf{Стек}: Java, YQL, YDB, Protobuf, Yandex.Monitoring, \href{https://habr.com/ru/companies/yandex/articles/520134/}{Apphost}, Python, Jira / Yandex.Tracker}}
\subsubsection{Я разрабатывал ленту новостей для \href{https://dzen.ru}{\underline{dzen.ru}}. Детально что я делал:}
\begin{itemize}
    \item Я создал \textbf{систему мониторинга и оповещения} для одного из сервисов для контроля его состояния и производительности
    \item Я запустил несколько \textbf{A/B тестов} которые позволили найти неиспользуемую пользователями функциональность
    \item Я добавил несколько \textbf{offheap protobuf полей} и прокинул их через API в другие сервисы
    \item Я реализовал \textbf{контейнер для виджетов} (погода, геопозиция, курсы валют и тд) для упрощения добавления новых виджетов
    \item Я также написал \textbf{автоматические Python-скрипты} для рефакторинга большого объёма кода (более 6 тыс. строк)
    \item Вдобавок к выполнению обычных задач, я \textbf{написал документацию} о системах мониторинга и их коммуникациях между разными сервисами Дзена
    \item Иногда я \textbf{организовывал всю продуктовую команду на Scrum StandUp}
\end{itemize}

\subsection{\href{https://polis.vk.company/}{\underline{Технополис}} $|$ {\normalfont\textit{Java Developer}} \hfill Февраль 2023 -- Июнь 2023 (5 месяцев)}
\subsubsection{\underline{\textbf{Стек}: Java, Spring (Boot), PostgreSQL, TamTam API, \href{https://robbi.ai/}{robbi.ai}}}
\subsubsection{Я разрабатывал \href{https://github.com/Sanerins/tamtam-one-coffee-bot}{\underline{выпускной проект <<RandomCoffee Bot>>}} в команде в соответствие с бизнес-кейсом компании \href{https://ok.ru/}{\underline{<<Одноклассники>>}} для улучшения социальной коммуникации внутри компании. Детально что я делал:}
\begin{itemize}
    \item Создал \textbf{интерфейсы и классы для взаимодействия с PostgreSQL}
    \item Добавил \textbf{сохранение истории статусов взаимодействий пользователей} для улучшения в дальнейшем поиска собеседников
    \item \textbf{Рефакторил код} для улучшения читаемости
\end{itemize}

\subsection{\href{https://www.spbstu.ru/}{\underline{СПбПУ}} $|$ {\normalfont\textit{Python Developer}} \hfill Февраль 2022 -- Июнь 2022 (5 месяцев)}
\subsubsection{\underline{\textbf{Стек}: Python (PIL), SQLite,  \href{https://github.com/python-telegram-bot/python-telegram-bot}{Telegram API}, Docker}}
\subsubsection{Я разрабатывал \href{https://github.com/PaaavelZ/FPA-pybot}{\underline{Telegram-бота}} в команде в соответствие с бизнес-кейсом компании \href{https://www.wbc-c.ru/}{\underline{<<Белорусская косметика>>}} для защиты медиа-контента от плагиата. Детально что я делал:}
\begin{itemize}
    \item Создал \textbf{интерфейсы и классы для взаимодействия с SQLite}
    \item Добавил \textbf{архивную обработку фотографий}
    \item \textbf{Подготовил документацию} по проекту
    \item \textbf{Взаимодействовал с заказчиком} и искал с ним общий язык для решения поставленных задач
\end{itemize}

\subsection{\href{https://www.spbstu.ru/}{\underline{Завод на базе СПбПУ}} $|$ {\normalfont\textit{Java Developer}} \hfill Октябрь 2021 -- Январь 2022 (4 месяца)}
\subsubsection{\underline{\textbf{Стек}: Java, Spring (Boot, Web, Data JPA, Security), HTML, CSS, JS, \href{https://deepsource.com/}{DeepSource}}}
\subsubsection{Я разрабатывал \href{https://github.com/martyn-fanclub/tracking-system}{\underline{систему слежения за сотрудниками предприятия}} в команде в соответствие с бизнес-кейсом \href{https://www.spbstu.ru/}{\underline{завода на базе СПбПУ}} для отслеживания прогресса работы сотрудников завода на станках. Детально что я делал:}
\begin{itemize}
    \item Реализовал \textbf{глобальный таймер} для отслеживания текущего времени в UI
    \item Имплементировал несколько \textbf{эндпоинтов} на серверной части
    \item \textbf{Взаимодействовал с заказчиком} и искал с ним общий язык для решения поставленных задач
\end{itemize}

\section{Projects}
%=================
\subsection{\href{https://polis.vk.company/}{\underline{Технополис}} $|$ {\normalfont\textit{Java Developer}} \hfill Февраль 2022 -- Декабрь 2022}
\subsubsection{\underline{\textbf{Стек}: Java, Cassandra, Lua, JUnit, \href{https://github.com/odnoklassniki/one-nio}{one-nio}, \href{https://github.com/giltene/wrk2}{wrk2}, \href{https://github.com/async-profiler/async-profiler}{async-profiler}}}
\subsubsection{Я разрабатывал \href{https://github.com/polis-vk/2022-highload-dht/tree/main/src/main/java/ok/dht/test/kovalenko}{\underline{мою собственную NoSQL БД}} как часть \href{https://education.vk.company/}{\underline{образовательного проекта VK}}, Cassandra была взята за прототип. Детально что я делал:}
\begin{itemize}
    \item В качестве фундамента БД добавил \textbf{персистентность} для сохранения данных, \textbf{Range-запросы} с бинарным поиском и \textbf{Compaction} для удаления старых данных
    \item В качестве сетевого взаимодействия я добавил \textbf{HTTP REST API} для получения данных по эндпоинтам, \textbf{асинхронный сервер} для разгрузки SelectorThread'ов, \textbf{шардирование} для увеличения горизонтальной масштабируемости, \textbf{репликацию} для обеспечения отказоустойчивости, \textbf{асинхронное взаимодействие} между кластерными нодами, \textbf{Range-запросы с HTTP Chunked Response} для разгрузки сети
    \item Также я активно профилировал свою БД, оптимизируя узкие места системы и получая развёрнутый фидбек от экспертов в этой области
\end{itemize}

\begin{comment}
\subsection{\href{https://github.com/PaaavelZ/tinkoff_bot}{\underline{Pet-project}} $|$ {\normalfont\textit{Python Developer}} \hfill June 2022 -- July 2022}
\subsubsection{\underline{\textbf{Stack}: Python, \href{https://github.com/python-telegram-bot/python-telegram-bot}{Telegram API}, SQLite, \href{https://github.com/Tinkoff/invest-python}{Tinkoff API}}}
\subsubsection{I developed \href{https://github.com/PaaavelZ/tinkoff_bot}{\underline{a Telegram financial bot}} for personal use. In more detail what did I do:}
\begin{itemize}
    \item I created \textbf{a SQLite database class interface} to organize and access all application data
    \item I implemented \textbf{the forming and sending aggregated reports} of \href{https://www.tinkoff.ru/invest/}{\underline{the Tinkoff Invest's}} daily trade statistics for chosen accounts as well as the Tinkoff Investment App but it is available only for a month officially
    \item I also added the \textbf{subscription's capabilities} to be able to independently write to the bot and receive personal reports from it
\end{itemize}
\end{comment}

\section{Skills}
%===============
\begin{description}
    \item[Технические] Java, Spring (Boot, Data JPA, Web, Security), PostgreSQL, Cassandra, Docker, Python, C++, Yandex.Monitoring, Jira
    \item[Языки] English - B1
\end{description}

\section{Reading List}
%===============
\begin{description}
    \item \href{https://www.amazon.com/Clean-Code-Handbook-Software-Craftsmanship/dp/0132350882}{\underline{Роберт Мартин <<Чистый код>>}}
\end{description}

\section{Certification \& Awards}
%===============================
\begin{enumerate}[itemsep=0pt]
    \item [2024] \href{https://disk.yandex.ru/i/_CuHJW-1k6qG8A}{\underline{Сертификат EF SET English B1}}
    \item [2023] \href{https://disk.yandex.ru/i/Ftcyy4kuYyXlEQ}{\underline{Сертификат выпускника Технополиса}}
\end{enumerate}

\end{document}
